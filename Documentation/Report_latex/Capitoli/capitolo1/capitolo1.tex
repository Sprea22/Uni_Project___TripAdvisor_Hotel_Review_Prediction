\chapter{Introduzione al Progetto}
\section{Motivazione}
Le recensioni scritte dagli utenti circa beni acquistati o servizi usufruiti hanno un elevato impatto sui fornitori di tali beni e servizi in termini di reputazione. Negli ultimi anni, la cosiddetta Brand Reputation, ha assunto un ruolo sempre più rilevante a livello decisionale per permettere di affrontare criticità relative a beni  e servizi commercializzati, ridurre la perdita di potenziale clientela ed infine mantenere un elevato potere competitivo all’interno del mercato.  Date quindi le opinioni espresse dai clienti sui vari social networks (Twitter, Facebook, TripAdvisor, etc…), è necessario quantificare tali opinioni per fornire indicatori di performance relativi ai beni/servizi offerti, per permettere decisioni correttive/strategiche. 

\section{Dati}
\begin{itemize}
\item Recensioni scritte in linguaggio naturale derivanti da TripAdvisor
\item Metadati: rating associati a singoli aspetti (rooms, location, service etc…)
\item Target: overall rating 
\end{itemize}


\begin{longtable}{l |c|}

\hline
	
Review ID = 1
<Author>moonkwean\\
<Content>Everything you need Stayed 9 days in Phoenix but only three nights at this hotel.
\\Wished we had booked our whole stay here. Staff very friendly and available. 
\\Any time we requested something it was attended to right away. Rooms clean.
\\Bonus is the breakfast, social time and the 3 computers in the lobby.
\\ Would definitely stay here again. Thanks David, Pamela and Mila.\\
<Date>Jan 18, 2006\\
<No. Reader>17\\
<No. Helpful>14\\
<Overall>4\\
<Value>5\\
<Rooms>4\\
<Location>-1\\
<Cleanliness>5\\
<Check in / front desk>-1\\
<Service>5\\
<Business service>-1\\


\hline
\end{longtable}
%%%%%%%%%%%%%%%%%%%%%%%%%%%%%%%%%%%%%%%%%%%%%%%%%%%%%%%%%%%%%%



